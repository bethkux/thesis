\chapter{More complicated chapter}
\label{chap:math}

After the reader gained sufficient knowledge to understand your problem in \cref{chap:refs}, you can jump to your own advanced material and conclusions.

You will need definitions (see \cref{defn:x} below in \cref{sec:demo}), theorems (\cref{thm:y}), general mathematics, algorithms (\cref{alg:w}), and tables (\cref{tab:z})\todo{See documentation of package \texttt{booktabs} for hints on typesetting tables. As a main rule, \emph{never} draw a vertical line.}. \Cref{fig:f,fig:g} show how to make a nice figure. See \cref{fig:schema} for an example of TikZ-based diagram. Cross-referencing helps to keep the necessary parts of the narrative close --- use references to the previous chapter with theory wherever it seems that the reader could have forgotten the required context. Conversely, it is useful to add a few references to theoretical chapters that point to the sections which use the developed theory, giving the reader easy access to motivating application examples.

\section{Example with some mathematics}
\label{sec:demo}

\begin{defn}[Triplet]\label{defn:x}
Given stuff $X$, $Y$ and $Z$, we will write a \emph{triplet} of the stuff as $(X,Y,Z)$.
\end{defn}

\newcommand{\Col}{\textsc{Colour}}

\begin{thm}[Car coloring]\label{thm:y}
All cars have the same color. More specifically, for any set of cars $C$, we have
$$(\forall c_1, c_2 \in C)\:\Col(c_1) = \Col(c_2).$$
\end{thm}

\begin{proof}
Use induction on sets of cars $C$. The statement holds trivially for $|C|\leq1$. For larger $C$, select 2 overlapping subsets of $C$ smaller than $|C|$ (thus same-colored). Overlapping cars need to have the same color as the cars outside the overlap, thus also the whole $C$ is same-colored.\todo{This is plain wrong though.}
\end{proof}

\begin{table}
% uncomment the following line if you use the fitted top captions for tables
% (see the \floatsetup[table] comments in `macros.tex`.
%\floatbox{table}[\FBwidth]{
\centering\footnotesize\sf
\begin{tabular}{llrl}
\toprule
Column A & Column 2 & Numbers & More \\
\midrule
Asd & QWERTY & 123123 & -- \\
Asd qsd 1sd & \textcolor{red}{BAD} & 234234234 & This line should be helpful. \\
Asd & \textcolor{blue}{INTERESTING} & 123123123 & -- \\
Asd qsd 1sd & \textcolor{violet!50}{PLAIN WEIRD} & 234234234 & -- \\
Asd & QWERTY & 123123 & -- \\
\addlinespace % a nice non-intrusive separator of data groups (or final table sums)
Asd qsd 1sd & \textcolor{green!80!black}{GOOD} & 234234299 & -- \\
Asd & NUMBER & \textbf{123123} & -- \\
Asd qsd 1sd & \textcolor{orange}{DANGEROUS} & 234234234 & (no data) \\
\bottomrule
\end{tabular}
%}{  % uncomment if you use the \floatbox (as above), erase otherwise
\caption{An example table.  Table caption should clearly explain how to interpret the data in the table. Use some visual guide, such as boldface or color coding, to highlight the most important results (e.g., comparison winners).}
%}  % uncomment if you use the \floatbox
\label{tab:z}
\end{table}

\begin{figure}
\centering
\includegraphics[width=.6\linewidth]{img/ukazka-obr02.pdf}
\caption{A figure with a plot, not entirely related to anything. If you copy the figures from anywhere, always refer to the original author, ideally by citation (if possible). In particular, this picture --- and many others, also a lot of surrounding code --- was taken from the example bachelor thesis of MFF, originally created by Martin Mareš and others.}
\label{fig:g}
\end{figure}

\begin{figure}
\centering
\tikzstyle{box}=[rectangle,draw,rounded corners=0.5ex,fill=green!10]
\begin{tikzpicture}[thick,font=\sf\scriptsize]
\node[box,rotate=45] (a) {A test.};
\node[] (b) at (4,0) {Node with no border!};
\node[circle,draw,dashed,fill=yellow!20, text width=6em, align=center] (c) at (0,4) {Ugly yellow node.\\Is this the Sun?};
\node[box, right=1cm of c] (d) {Math: $X=\sqrt{\frac{y}{z}}$};
\draw[->](a) to (b);
\draw[->](a) to[bend left=30] node[midway,sloped,anchor=north] {flow flows} (c);
\draw[->>>,dotted](b) to[bend right=30] (d);
\draw[ultra thick](c) to (d);

\end{tikzpicture}
\caption{An example diagram typeset with TikZ. It is a good idea to write diagram captions in a way that guides the reader through the diagram. Explicitly name the object where the diagram viewing should ``start''. Preferably, briefly summarize the connection to the parts of the text and other diagrams or figures. (In this case, would the tenative yellow Sun be described closer in some section of the thesis? Or, would there be a figure to detail the dotted pattern of the line?)}
\label{fig:schema}
\end{figure}

\begin{algorithm}
\begin{algorithmic}
\Function{ExecuteWithHighProbability}{$A$}
	\State $r \gets$ a random number between $0$ and $1$
	\State $\varepsilon \gets 0.0000000000000000000000000000000000000042$
	\If{$r\geq\varepsilon$}
		\State execute $A$ \Comment{We discard the return value}
	\Else
		\State print: \texttt{Not today, sorry.}
	\EndIf
\EndFunction
\end{algorithmic}
\caption{Algorithm that executes an action with high probability. Do not care about formal semantics in the pseudocode --- semicolons, types, correct function call parameters and similar nonsense from `realistic' languages can be safely omitted. Instead make sure that the intuition behind (and perhaps some hints about its correctness or various corner cases) can be seen as easily as possible.}
\label{alg:w}
\end{algorithm}

\section{Extra typesetting hints}

Do not overuse text formatting for highlighting various important parts of your sentences. If an idea cannot be communicated without formatting, the sentence probably needs rewriting anyway. Imagine the thesis being read aloud as a podcast --- the storytellers are generally unable to speak in boldface font.

Most importantly, do \underline{not} overuse bold text, which is designed to literally \textbf{shine from the page} to be the first thing that catches the eye of the reader. More precisely, use bold text only for `navigation' elements that need to be seen and located first, such as headings, list item leads, and figure numbers.

Use underline only in dire necessity, such as in the previous paragraph where it was inevitable to ensure that the reader remembers to never typeset boldface text manually again.

Use \emph{emphasis} to highlight the first occurrences of important terms that the reader should notice. The feeling the emphasis produces is, roughly, ``Oh my --- what a nicely slanted word! Surely I expect it be important for the rest of the thesis!''

Finally, never draw a vertical line, not even in a table or around figures, ever. Vertical lines outside of the figures are ugly.

\chapter{Acknowledgments}

\section{Inspirations}
Michaela Štolová's \textit{UCube - mobilní aplikace pro speedcubing (2020)} \cite{Stolova2020}, Vilém Gutvald's \textit{Tower Defense Game with Procedurally Generated Content and Rogue-like Elements (2024)} \cite{Gutvald2024} and Martin Vejbora's \textit{Carty – Arcade Racing Game (2020)} \cite{Vejbora2020}  bachelor theses served as references for shaping the structure of this thesis, given their relevance to the topic. 

\section{Use of AI}
ChatGPT-4 was used while working on this thesis to help find better wording and make sentences sound more natural. It was not used to generate any parts of the text directly. It also helped with debugging some parts of the project’s code. 

\section{External Assets}
This project uses assets such as sprites and dialogue text from the classic titles \textit{The Secret of Monkey Island} (1990) and \textit{Beneath a Steel Sky} (1994). The use of these materials is strictly non-commercial and intended solely for educational purposes within the scope of this framework. The creators of TaleCraft do not endorse or encourage unauthorized use of copyrighted content. Instead, we strongly encourage readers to support the original developers by purchasing the official releases. A sequel to \textit{Beneath a Steel Sky} named \textit{Beyond a Steel Sky} is available on Steam \cite{Beyond-a-Steel-Sky}, and \textit{The Secret of Monkey Island} has been rereleased as a modern remake and also can be found on Steam \cite{TSoMI-steam}. 

The remaining assets used in the project are licensed under Creative Commons and include proper attribution in the folders where they are located. 
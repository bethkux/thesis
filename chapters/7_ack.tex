\chapter{Acknowledgments}

\section{Inspirations}
Michaela Štolová's \textit{UCube - mobilní aplikace pro speedcubing (2020)} \cite{Stolova2020}, Vilém Gutvald's \textit{Tower Defense Game with Procedurally Generated Content and Rogue-like Elements (2024)} \cite{Gutvald2024} and Martin Vejbora's \textit{Carty – Arcade Racing Game (2020)} \cite{Vejbora2020} bachelor theses have been used as a source for inspiration on the thesis structure and content due to the similarity of the subject matter.

\section{Use of AI in the thesis and in the project}
The AI tool ChatGPT-4 has been used during the writing of this thesis to help find a suitable word or expression for sentences to avoid unnatural phrasing. No text has been generated by it directly. It has been used to try to debug some parts of the project's code.

\section{External Assets}
This project incorporates assets such as sprites and dialogue text from the classic titles \textit{The Secret of Monkey Island} (1990) and \textit{Beneath a Steel Sky} (1994). The use of these materials is strictly non-commercial and intended solely for educational purposes within the scope of this framework. The creators of TaleCraft do not endorse or encourage unauthorized use of copyrighted content. Instead, we strongly encourage readers to support the original developers by purchasing the official releases. A sequel to \textit{Beneath a Steel Sky} is available on Steam, and \textit{The Secret of Monkey Island} has been re-released as a modern remake. 

All other assets used in the project are under creative commons license and are credited in the folders they are stored in.  